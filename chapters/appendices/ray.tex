\label{Ray}

Appendix B...

\section{Calculating the Camera Ray}

For each pixel, the process to calculate the camera ray (defined in Section \ref{Rendering}) involves transforming screen coordinates to a normalized space, and then determining the corresponding direction in the world space.

\paragraph{Screen to normalized space}
To make the rendering resolution-independent, each pixel's screen coordinates are converted to a normalized space. Let $\mathbf{P}$ be the screen coordinates and $\mathbf{res}$ be the resolution. The normalized coordinates $\mathbf{N}$ are given by:

\begin{equation}
   \mathbf{N} = \frac{2 \cdot \mathbf{P} - \mathbf{res}}{\min(\mathbf{res}.x, \mathbf{res}.y)}
\end{equation}

\paragraph{World position calculation}
To find the world position for each pixel, the camera parameters are used. Let $\mathbf{C}$ be the camera position, $\mathbf{f}, \mathbf{r}, \mathbf{u}$ be the forward, right and up directions respectively, and $f$ be the focal length. Here, $f$ is the distance in world coordinates between the camera $\mathbf{C}$ and the projection plane, influencing the field of view. The world position $\mathbf{W}$ is computed as:

\begin{equation}
   \mathbf{W} = \mathbf{C} + \text{normalize}(\mathbf{f}) \cdot f + \mathbf{N}.x \cdot \text{normalize}(\mathbf{r}) + \mathbf{N}.y \cdot \text{normalize}(\mathbf{u})
\end{equation}

\paragraph{Deriving the camera ray}
The direction of the camera ray $\mathbf{r_c}$ is the normalized vector from the camera position to the world position of the pixel. It is expressed as:

\begin{equation}
   \mathbf{r_c} = \text{normalize}(\mathbf{W} - \mathbf{C})
\end{equation}

These steps establish the direction of each ray from the camera to the pixel in world space, setting the foundation for ray marching.

\section{Calculating the Sun Ray}
\label{Sun Ray}

In this project, the \textbf{sun ray}, $\mathbf{r_s}$, is the normalized vector from a point in the world, towards the sun. The sun is treated as a directional light source due to its immense distance from Earth, implying that the sun rays are parallel at every point.

The sun ray's direction is represented using spherical coordinates, which offer a convenient method to define the direction in 3D space using only two parameters: azimuth ($\phi$) and elevation ($\theta$). These parameters are adjustable in the UI, allowing run-time changes to the sun ray. $\mathbf{r_s}$ is computed as follows:

\begin{equation}
    \mathbf{r_s} = \text{normalize}\left( \begin{bmatrix} \sin(\theta) \cos(\phi) \\ \cos(\theta) \\ \sin(\theta) \sin(\phi) \end{bmatrix} \right)
\end{equation}

This vector is critical for lighting calculations in the rendering process.
